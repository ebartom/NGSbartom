\documentclass[11pt]{report}

\usepackage{times,fullpage,graphicx,amsmath}
\usepackage[pdfborder={0 0 0}]{hyperref}

\usepackage{subfigure}

\newcommand{\supth}{{\mathrm{th}}}

\title{RSEG Manual}
\author{Qiang Song \and Andrew Smith}

\begin{document}
\maketitle

\chapter{Quick Start}
\label{chap:quick-start}

The RSEG software package is aimed to analyze ChIP-Seq data,
especially for identifying genomic domains marked by diffusive histone
modification markers, such as H3K36me3 and H3K9me3. It can work with
or without control sample. It can be used to find regions with
differential histone modifications patterns, either comparsion between
two cell types or between two kinds of histone modifications.

\section{Installation}
\label{sec:install}

\subsection*{Download}

RSEG, including pre-compiled binary files and source code, is
available at \url{http://smithlab.cmb.usc.edu/histone/rseg/}.

\subsection*{System Requirement}

RSEG runs on Linux and Mac OS operating system. The GNU Compilation
Collection (GCC) is necessary if you want to compile by yourself.

\subsection*{Installation}

If you compile from source code, download the source code and
decompress it with
\begin{verbatim}
 $ tar exfz rseg-v0.0.0.tar.gz
\end{verbatim}
%
Enter the rseg directory, run
\begin{verbatim}
 $ make && make install
\end{verbatim}
%
If complied successfully, the executable files are located in \textbf{rseg/bin}.


\section{Using RSEG}
\label{sec:usage}

Here are some examples using RSEG. For complete usage, type \textbf{rseg
 --help} or go to the Section \ref{sec:usage-detail}.

\subsection{Single Sample Analysis}
\label{sec:use-rseg}

\paragraph{Basic usage:} 
To find the functinoal domains for certain histone modification
markers without control sample, use the program \textit{rseg}. Use -o the
directory where the output is writen to; use -c to specify the file
listing the size of chromosomes; user -i to specif number of
iterations for Baum training. The last parameter is a BED file that
contains mapped reads in sorted order. You can add -v to show more
information.
\begin{verbatim}
$ rseg -c mouse-mm9-size.bed -o $PWD -i 20  -v  ES.K36.bed
\end{verbatim}

\paragraph{deadzone correction:}
Using deadzones correction may signigicanty improve the quality of
identified domains. You can give an BED file containing the location
of deadzones with -d option. Use the appropriate genome assembly and
read length (see Section \ref{sec:deadzone} for more information about
deadzones)
\begin{verbatim}
$ rseg -c mouse-mm9-size.bed -o $PWD -i 20 -d deadzone-mm9-k27.bed 
ES.K36.bed -v
\end{verbatim}


\subsection{Two sample analysis}
\label{sec:use-rseg-diff}


\paragraph{Use a control sample:}
To work with a control sample, use \textit{rseg-diff} with the option
\textbf{-mode 2}. Most of the options above, such as bin size, deadzone, etc,
can be used similarly with \textit{rseg}. \textit{rseg-diff} assumes that first
input file is test sample and the second input file is control sample.
\begin{verbatim}
$ rseg-diff -c mouse-mm9-size.bed -o $PWD -i 20 -v -mode 2
-d deadzone-mm9-k27.bed ES.K36.bed ES.WCE-control.bed
\end{verbatim}

\paragraph{Compare two test samples:}
To compare the histone modification pattern of two sample, use
\textit{rseg-diff} with \textbf{-mode 3}. Most of the options above, such as bin size,
deadzone, etc, can be used similarly with \textit{rseg}.
\begin{verbatim}
$ rseg-diff -c human-hg18-size.bed -o $PWD -i 20 -v -mode 3 
-d deadzone-hg18-k25.bed CD133.K36.bed CD36.K36.bed
\end{verbatim}

\section{File Format}
\label{sec:file-format}

RSEG works with
\href{http://genome.ucsc.edu/FAQ/FAQformat.html#format1}{BED} used in
UCSC Genome Browser as both input format and output format. If you use
alternative mapping format produced from ELAND, MAQ, bowtie, etc, you
need first to convert it to BED format. Hopefully you know how :-),
otherwise you may would like to look at the
\href{http://sourceforge.net/apps/mediawiki/vancouvershortr/index.php?title=ConvertToBed}
{ConvertToBed} utility provided by
\href{http://sourceforge.net/projects/vancouvershortr/}{Vancouver
  Short Read Analysis Package}.


\subsection{Input file format}
\label{sec:input-file}

\paragraph{Mapped read file:} The input file containing mapped reads
is of the format of a 6-column
\href{http://genome.ucsc.edu/FAQ/FAQformat.html#format1}{BED}
file. Further the reads in input file should be sorted (see
\ref{sec:sortbed} for how to sort reads file).

\paragraph{Chromosome size file:} Both \textit{rseg} and
\textit{rseg-diff} requires an input file that specifies the size of
chromosomes. This file is a 3-column
\href{http://genome.ucsc.edu/FAQ/FAQformat.html#format1}{BED}
file. The $1^{st}$ specifies the chromosome name, the $2^{nd}$ column
specifies the start of chromosome and the $3^{rd}$ column the end of
column. See \href{http://smithlab.cmb.usc.edu/histone/software/}{RSEG
  Website} for a list of chromosome size files for common model
organisms. For other organisms, you can go to UCSC Table Browser
(\url{http://genome.ucsc.edu/cgi-bin/hgTables?command=start}), select
the desired organism and assembly and then choose group: All Tables
and table: chrominfo. Table Browser will return the name and sizes of
all chromosomes, from which you can manually compile a BED used as
chromosome size file for \textit{rseg}.

\paragraph{Deadzone files:} Both \textit{rseg} and \textit{rseg-diff}
recommend the use of a deadzone file suitable for the given genome
assembly and read length. This file is a 3-column
\href{http://genome.ucsc.edu/FAQ/FAQformat.html#format1}{BED}
file. Each line shows the location of a deadzone. See
\href{http://smithlab.cmb.usc.edu/histone/software}{RSEG Website} for
a list of deadzone files for common model organisms and selected read
length or use the \textit{deadzone} program to compute deadzones
(Section ~ \ref{sec:deadzone}).
 
\subsection{RSEG output files}
\label{sec:rseg-output}

Depending on the options specified, \textit{rseg} may produce up to five output
files. Suppose your input BED file is \textit{ES.K36.bed}, these five output
files are \textit{ES.K36-domains.bed}, \textit{ES.K36-scores.wig},
\textit{ES.K36-boundaries.bed}, \textit{ES.K36-boundary-scores.wig},
and \textit{ES.K36-counts.bed}.

\textbf{ES.K36-domains.bed} is a 7-column
\href{http://genome.ucsc.edu/FAQ/FAQformat.html#format1}{BED} file
(Table~\ref{tab:format-domain}). Each line shows the information of an
epigenomic domain. The $4^{th}$ column denotes the state of each
domain: ENRICHED.  The $5^{th}$ column gives the average read count in
the domain. The $6^{th}$ column is the sum of posterior scores of all
bins within this domain; it measures both the quality and size of the
domain. The $7^{th}$ does not have specific meaning.

\begin{table}[th]
  \centering
  \begin{tabular}{c c c c c c c}
Column 1 & Column 2 & Column 3 &  Column 4 & Column 5 &  Column 6  &
Column 7 \\
\hline
Chromosome  & Start & End & Domain State &  Avg Count & Domain Score &  Strand \\
\hline
chr1 &   744100  &780500  &ENRICHED        &9.57089 &11.9706 &+ \\
chr1 &   870100  &882700  &ENRICHED        &13.0536 &17.455  &+ \\
chr1 &   1026900 &1039500 &ENRICHED        &7.43915 &10.401  &+ \\
chr1 &   1141700 &1154300 &ENRICHED        &8.85827 &16.3838 &+ \\
\ldots & \ldots &\ldots &\ldots &\ldots &\ldots & \ldots\\ 
\hline
  \end{tabular}
  \caption{ES.K36-domains.bed: Domain output file format}
  \label{tab:format-domain}
\end{table}

\textbf{ES.K36-scores.wig} is a 4-column
\href{http://genome.ucsc.edu/goldenPath/help/bedgraph.html}{BedGraph}
file (Table~\ref{tab:format-bin-score}). Each line shows the posterior
probability of that bin being in the enriched (foreground) state. This
file can be used to visulize the status of each bin in UCSC Genome
Browser.

\begin{table}[th]
  \centering
  \begin{tabular}{c c c c}
Column 1 & Column 2 & Column 3 &  Column 4  \\
\hline
Chromosome  & Start & End & Posterior Prob. \\
\hline
chr1&    3000000& 3001752& 0.999252\\
chr1&    3001752& 3003504& 0.999901\\
chr1&    3003504& 3005256& 0.999961\\
chr1&    3015768& 3017520& 0.999868\\
\ldots & \ldots &\ldots &\ldots \\ 
\hline
  \end{tabular}
  \caption{ES.K36-scores.wig: Bin posterior score  output file format}
  \label{tab:format-bin-score}
\end{table}


\textbf{ES.K36-boundaries.bed} is a 6-column
\href{http://genome.ucsc.edu/FAQ/FAQformat.html#format1}{BED} file
(Table~\ref{tab:format-boundary}). Each line represents a
boundary. The $4^{th}$ column gives more information about this
boundary: after ``B'', it gives in order the upper limit of the size
of this boundary in bins, the location of boundary peak and the
posterior transisiton probability at the peak. The $5^{th}$ gives the
posterior transisiton probability that a single transisiton occurs
within this boundary.
\begin{table}[th]
  \centering
  \begin{tabular}{c c c c c c }
    Column 1 & Column 2 & Column 3 &  Column 4 & Column 5 &  Column 6 \\
    \hline
    Chromosome  & Start & End & Boundary Peak &  Posterior Transisiton &  Strand \\
    \hline
chr1&    5153208& 5154960& B:1:5153208:0.7345 &     0.7345 & + \\
chr1&    9923904& 9925656& B:1:9923904:0.705447&    0.705447&        +\\
chr1&    9934416& 9936168& B:1:9934416:0.87405 &    0.87405& +\\
    \ldots & \ldots &\ldots &\ldots &\ldots &\ldots \\ 
    \hline
  \end{tabular}
  \caption{ES.K36-boundaries.bed: Domain Boundaries output file format}
  \label{tab:format-boundary}
\end{table}

\textbf{ES.K36-boundary-scores.wig} is is a 4-column
\href{http://genome.ucsc.edu/goldenPath/help/bedgraph.html}{BedGraph}
file (Table~\ref{tab:format-bound-score}). Each line gives the posterior
transition probability at that bin.

\begin{table}[th]
  \centering
  \begin{tabular}{c c c c}
    Column 1 & Column 2 & Column 3 &  Column 4  \\
    \hline
    Chromosome  & Start & End & Posterior Transition Prob. \\
    \hline
    chr1&    7078001& 7079001& 0.013952 \\
    chr1&    7079001& 7080001& 0.109364 \\
    chr1&    7080001& 7081001& 0.859525 \\
    chr1&    7081001& 7082001& 0.014624 \\
    \ldots & \ldots &\ldots &\ldots \\ 
    \hline
  \end{tabular}
  \caption{ES.K36-boundary-scores.wig: posterior transition score output file}
  \label{tab:format-bound-score}
\end{table}

\textbf{ES.K36-counts.bed} is a 6-column
\href{http://genome.ucsc.edu/FAQ/FAQformat.html#format1}{BED}
file (Table~\ref{tab:format-bin}). Each line represents a bin. The $4^{th}$, $5^{th}$ and $6^{th}$
give the number of reads, the non-deadzone proportion and the state in
this bin.

\begin{table}[th]
  \centering
  \begin{tabular}{c c c c c c }
    Column 1 & Column 2 & Column 3 &  Column 4 & Column 5 &  Column 6 \\
    \hline
    Chromosome  & Start & End & Read Count & Non-deadzone proportion &
    State Label \\
    \hline
chr1&    3000000& 3001752& 2&       0.938927&        0 \\
chr1&    3001752& 3003504& 2&       0.918379&        0 \\
chr1&    3003504& 3005256& 0&       0.680365&        0 \\
chr1&    3015768& 3017520& 3&       0.550228&        0 \\
    \ldots & \ldots &\ldots &\ldots &\ldots &\ldots \\ 
    \hline
  \end{tabular}
  \caption{ES.K36-counts.bed: Bin statistics output file format}
  \label{tab:format-bin}
\end{table}

\subsection{RSEG-DIFF output files}
\label{sec:rseg-output}

Depending the options specified, \textit{rseg-diff} may produce up to
five output files. Suppose your input BED file is \textit{ES.K36.bed}
and your input control file is \textit{WCE.bed}, these five output
files are \textit{ES.K36-WCE-domains.bed},
\textit{ES.K36-WCE-scores.wig}, \textit{ES.K36-WCE-boundaries.bed},
\textit{ES.K36-WCE-boundary-scores.wig}, and 
\textit{ES.K36-WCE-counts.bed}. These files are similar to those
output from \textit{rseg} with the difference explained below.

\textbf{ES.K36-WCE-domains.bed} If you use \textit{rseg-diff} with the
option \textbf{-mode 2}. The domain file format is similar to that
specified in Table~\ref{tab:format-domain}. The $4^{th}$ column gives
domain state, where ENRICHED means the domain is enriched relative to
the control. The $5^{th}$ column gives average read count difference
in that domain (test sample subtracted by control sample).

\textbf{CD133.K36-CD36-domains.bed} If you use \textit{rseg-diff} with
the option \textbf{-mode 3}, the domain output file format is shown in
Table~\ref{tab:format-domain-diff}. The $4^{th}$ column gives the
domain state SAMPLE-I-ENRICHED means the histone in Sample I is
hyper-modified relative to that in Sample II, and SAMPLE-II-ENRICHED
means the histone in Sample I is hypo-modified relative to that in
Sample II.  The $5^{th}$ column gives the average read count
difference in that domain.

\begin{table}[th]
  \centering
  \begin{tabular}{c c c c c c c}
Column 1 & Column 2 & Column 3 &  Column 4 & Column 5 &  Column 6  &
Column 7 \\
\hline
Chromosome  & Start & End & Domain State &  Avg Count Diff.& Domain Score &  Strand \\
\hline
chr1&    1790153& 1800865& SAMPLE-II-ENRICHED &     -5.51454   &     11.2231& + \\
chr1&    1978025& 1987913& SAMPLE-I-ENRICHED &      6.87003& 7.07664& + \\
chr1&    1996977& 2000273& SAMPLE-I-ENRICHED &      11.9379 &3.7683 & + \\
\ldots & \ldots &\ldots &\ldots &\ldots &\ldots & \ldots\\ 
\hline
  \end{tabular}
  \caption{CD133.K36:CD36-domains.bed: Domain output file  by rseg-diff mode 3}
  \label{tab:format-domain-diff}
\end{table}


\textbf{ES.K36-WCE-scores.wig} is a 4-column
\href{http://genome.ucsc.edu/goldenPath/help/bedgraph.html}{BedGraph}
file (Table~\ref{tab:format-bin-score}). Each line shows the posterior
probability of that bin being in the enriched (foreground) state. This
file can be used to visulize the status of each bin in UCSC Genome
Browser.

\textbf{CD133.K36-CD36-scores.wig} If you use \textit{rseg-diff} with
the option \textbf{-mode 3}, for example, to compare H3K36me3 profile
between CD133 and CD36 cells, the score output file is a five-column
BED file (Table~\ref{tab:format-score-diff}). The $4^{th}$ column
gives posterior probability that the bin is is hyper-modified in
Sample I compared to Sample II, and the $5^{th}$ column gives the
posterior probability that the bin is hypo-modified in Sample I
compared to Sample II. The posterior probability that the bin does not
change between the two samples is obtained by subtracting the $4^{th}$
and $5^{th}$ columns from 1.0.

\begin{table}[th]
  \centering
  \begin{tabular}{c c c c c }
Column 1 & Column 2 & Column 3 &  Column 4 & Column 5  \\
\hline
Chromosome  & Start & End &  Posterior scores & Posterior scores \\
\hline
chr1 &   1316000& 1316700& 0.000348498&     0.276782 \\
chr1 &   1316700& 1317400& 0.000521605&     0.411373 \\
chr1 &   1317400& 1318100& 0.00186753 &     0.900723 \\
chr1 &   1318100& 1318800& 0.00254065 &     0.914996 \\
chr1 &   1318800& 1319500& 0.00228736 &     0.910634 \\
chr1 &   1320200& 1320900& 0.00330582 &     0.936304 \\
\ldots & \ldots &\ldots &\ldots &\ldots \\ 
\hline
  \end{tabular}
  \caption{\textit{rseg-diff} posterior probability  output with running mode 3}
  \label{tab:format-score-diff}
\end{table}

\textbf{ES.K9-WCE-boundaries.bed} and
\textbf{ES.K36-WCE-boundary-scores.wig} are of the same format as
in \textit{rseg}.

\textbf{ES.K9-WCE-counts.bed} is a 7-column
\href{http://genome.ucsc.edu/FAQ/FAQformat.html#format1}{BED} file
(Table~\ref{tab:format-bin-diff}). Each line represents a bin. The $4^{th}$,
$5^{th}$, $6^{th}$ and $7^{th}$ give the number of reads in Sample I, the number
of reads in Sample II, the non-deadzone proportion and the state label for this
bin.

\begin{table}[th]
  \centering
  \begin{tabular}{c c c c c c c }
    Column 1 & Column 2 & Column 3 &  Column 4 & Column 5 &  Column 6
    & Column 7\\
    \hline
    Chromosome  & Start & End & Read Count & Read Count & Non-deadzone proportion &
    State Label \\
    \hline
chr1&    3000000& 3001752& 2&       0&       0.938927&        0\\
chr1&    3001752& 3003504& 2&       0&       0.918379&        0\\
chr1&    3003504& 3005256& 0&       0&       0.680365&        0\\
    \ldots & \ldots &\ldots &\ldots &\ldots &\ldots & \ldots \\ 
    \hline
  \end{tabular}
  \caption{ES.K9:WCE-counts.bed: Bin statistics output file format}
  \label{tab:format-bin-diff}
\end{table}


\chapter{RSEG in Detail}
\label{chap:rseg-manual}

\section{Installation}
\label{sec:install}

\subsection*{Download}

RSEG, including pre-compiled binary files and source code, is
available at \url{http://smithlab.cmb.usc.edu/histone/rseg/}.

\subsection*{System Requirement}

RSEG runs with the Linux system and Mac OS. You will also need GNU
Compilation Collection (GCC) if you want to compile by yourself.

\subsection*{Installation}

If you would like to compile from source code, download the source
code and decompress it with
\begin{verbatim}
 $ tar exfz rseg-v0.0.0.tar.gz
\end{verbatim}
%
Enter the rseg directory, run
\begin{verbatim}
 $ make && make install
\end{verbatim}
If complied successfully, the executable files are located in \textbf{rseg/bin}.

\section{Detailed Usage}
\label{sec:usage-detail}

This section explains in detail the usage and options for
\textit{rseg} and \textit{rseg-diff}.

\subsection{rseg}
\label{sec:rseg-detail}

\textit{rseg} is used to find histone modification domains from a
single test sample. 

\begin{description}
\item[Generic information]
\item[-help] Print a usage message briefly summarizing these
  command-line options and basic usage, then exit.
\item[-v, -verbose] Print more information when the program is running
\item[Options to format output]
\item[-o, -output-dir] This option gives the output directory for
  \textit{rseg}. \textit{rseg} write output files to current working
  directory by default.
\item[-boundary] This option enables the program to compute the
  properties of domain boundaries and output the result
\item[-tracks] This option enables the program to output the
  posterior probabilities and posterior transition probabilities bin
  by bin in two separate files
\item[-read-counts-requested] This option enables the program to
  output a file containing read counts for each bin.  
\item[-name] Common prefix to file name of the output
  files. Default value is obtained by truncating extension name from
the input file
\item[Required input files and options]
\item[input file] This file contains mapped reads from a
  ChIP-seq experiment and should be sorted.
\item[-c, -chrom] A BED file specifies the size of chromosomes for
  analysis
\item[-d, -deadzone-file] This options specifies the name of deadzone
  file
\item[Options to fine tune the method]
\item[-i, -iteration] The maximum number of iterations for HMM
  training
\item[-b, -bin-size] An integer to specify the size of bins used in
  the program. Larger value speeds up the computation but may reduce
  the resolution of the domains. The default value is computed based
  on total read counts and the effective genome size.
\item[-bin-size-step] Intial bin size when reading in raw reads
  (default 50bp). The bigger this value, the less memory usage
\item[-Waterman] If the -bin-size option is not specified, use
  Waterman's asymptotic formula to select bin size
\item[-Hideaki] If the -bin-size option is not specified, use
  Hideaki's asymptotic formula to determine bin size
\item[-Hideaki-emp] If the -bin-size option is not specified, using
  Hideaki's empirical method to select bin size. This is the default
  method.
\item[-smooth] This option indicates whether the rate curve for bin
  size selection is smooth. By default it is true. However when
  analyzing more localized marks, you may want to use option to change
  the default settings
\item[-max-deadzone-prop] Maximum deadzone proportion allowed for
  retained bins
\item[-not-remove-jackpot] Do not remove duplicate reads
\item[-s, -domain-size] Expected size of domain (Default 20000) 
\item[-S, -desert-size] This option gives an integer value so that if
  the size of a deadzone is larger than this value, the deadzone is
  ignored from subsequent analysis
\item[-F, -fg] The emission distribution used in the program to model
  read counts. Possible values are \textbf{nbd} (negative binomial
  distribution) and \textbf{pois} (Poisson distribution). Default
  value is \textbf{nbd}. Poisson distribution is less accurate but
  faster.  The default value is \textbf{nbd}.
\item[-B, -bg] Same as \textbf{-F, -fg} 
\item[-P, -posterior] This option enables the program use posterior
  decoding instead of Viterbi decoding. The program use posterior
  decoding by default
\item[-posterior-cutoff] Posterior threshold for signigicant
  bins. Possible values range is [0.5, 1.0). The large this value is,
  the more significant the identified domains are
\item[-undef-region-cutoff] The minimum size of an undetermined region
\item[-cdf-cutoff] Possible values is (0, 1.0). The large this value
  is, the more significant the identified domains are. This value is
  the minimum value that accumulative probability that a random
  varible from the foreground distribution if smaller than the mean
  read count for.
\end{description}

\subsection{rseg-diff}
\label{sec:rseg-diff-detail}

\textit{rseg-diff} can be used in two ways: first, it is used to find
histone domains by using both a test sample and a control
sample. Second, it is used to find domains with different signals
either between two histone marks in the same cell type or between two
cell types with the same histone modifcation. 

\begin{description}
\item[Generic information]
\item[-help] Print a usage message briefly summarizing these
  command-line options and basic usage, then exit.
\item[-v, -verbose] Print more information when the program is running
\item[Options to format output]
\item[-o, -output-dir] This option gives the output directory for
  \textit{rseg}. \textit{rseg} write output files to current working
  directory by default.
\item[-boundary] This option enables the program to compute the
  properties of domain boundaries and output the result
\item[-tracks] This option enables the program to output the
  posterior probabilities and posterior transition probabilities bin
  by bin in two separate files
\item[-read-counts-requested] This option enables the program to
  output a file containing read counts for each bin.  
\item[-name] Common prefix to file name of the output
  files. Default value is obtained by truncating extension name from
the input file
\item[Required input files and options]
\item[input files] \textit{rseg-diff} requires two input files. In
  Mode 2, these two files are a input file and a control file. In Mode
  3, this two files are from two samples  
\item[-c, -chrom] A BED file specifies the size of chromosomes for
  analysis
\item[Options to fine tune the method]
\item[-m, -mode] This option specifies the mode the program is used
  for. Possible values are \textbf{2} and \textbf{3}. Mode 2 is used
  for analysis with a test sample and a control sample and mode 3 is
  used for analysis with two test samples.
\item[-i, -iteration] The maximum number of iterations for HMM
  training
\item[-b, -bin-size] An integer to specify the size of bins used in
  the program. Larger value speeds up the computation but may reduce
  the resolution of the domains. The default value is computed based
  on total read counts and the effective genome size.
\item[-bin-size-step] Intial bin size when reading in raw reads
  (default 50bp). The bigger this value, the less memory usage
\item[-Waterman] If the -bin-size option is not specified, use
  Waterman's asymptotic formula to select bin size
\item[-Hideaki] If the -bin-size option is not specified, use
  Hideaki's asymptotic formula to determine bin size
\item[-Hideaki-emp] If the -bin-size option is not specified, using
  Hideaki's empirical method to select bin size. This is the default
  method.
\item[-smooth] This option indicates whether the rate curve for bin
  size selection is smooth. By default it is true. However when
  analyzing more localized marks, you may want to use option to change
  the default settings
\item[-max-deadzone-prop] Maximum deadzone proportion allowed for
  retained bins
\item[-not-remove-jackpot] Do not remove duplicate reads
\item[-s, -domain-size] Expected size of domain (Default 20000) 
\item[-S, -desert-size] This option gives an integer value so that if
  the size of a deadzone is larger than this value, the deadzone is
  ignored from subsequent analysis
\item[-F, -fg] The emission distribution used in the program to model
  read count difference. Possible values are \textbf{nbdiff} (NBDiff
  distribution), \textbf{skel} (Poisson distribution) and
  \textbf{gauss} (Gaussian distribution). The default value is
  \textbf{nbdiff}.The other two distributions may be less accurate but
  faster.
\item[-B, -bg] Same as \textbf{-F, -fg} 
\item[-P, -posterior] This option enables the program use posterior
  decoding instead of Viterbi decoding. The program use posterior
  decoding by default
\item[-posterior-cutoff] Posterior threshold for signigicant
  bins. Possible values range is [0.5, 1.0). The large this value is,
  the more significant the identified domains are
\item[-cdf-cutoff] Possible values is (0, 1.0). The large this value
  is, the more significant the identified domains are. This value is
  the minimum value that accumulative probability that a random
  varible from the foreground distribution if smaller than the mean
  read count for.
\item[-undef-region-cutoff] The minimum size of an undetermined region
\end{description}


\section{Utilities}
\label{sec:utilities}

We provide the following utilities together with \textit{rseg} for
analyzing epigenomic domains.

\subsection{Sort read files}
\label{sec:sortbed}

\textit{rseg} requires the input read files are sorted, which can be
done with standard UNIX \textit{sort} tool as following:
\begin{verbatim}
$ export LC_ALL=C
$ sort -k1,1 -k3,3n -k2,2n -k6,6 -o sorted.bed input.bed
\end{verbatim}
Note that we need to set the locale of the shell environment to the C
programming language locale. 

\subsection{deadzone}
\label{sec:deadzone}
The \textit{deadzone} program in RSEG software package is used to compute
unmappable regions given genome assembly and read length. You need first to
download the genome sequence of the genome in fasta format from
\href{http://hgdownload.cse.ucsc.edu/downloads.html}{UCSC Genome Browser
  Download}. Suppose the fasta files containing the sequence for mouse mm9 is
located at mm9/. You can compute unmappable regions for 32bp reads by running
the following command. 

\begin{verbatim}
$ deadzone -s fa -k 32 -o deadzones-mm9-k32.bed  mm9
\end{verbatim}

Optionally, you may change the -prefix option to adjust memory usage. The option
specifies the length of the prefix when the deadzone program enumerates all
possible kmers. The larger this option is, the more memory the program consumes
and the faster the program runs. The default value is $5$.

\section{Computional complexity}
\label{sec:comp-compl}
The computation resource usage by RSEG depends on several factors,
such as analysis type (single sample or double sample), binning size
(depends on reads number and genome size), number of interation during
HMM training and the number of bins used for training. The following
table lists estimates of time and memory requirement in a typical
analysis. 

We ran RSEG in a single computational node which has Intel(R) Xeon(R)
E5420 @ 2.50GHz CPU and 12010MB RAM. We use CentOS with Linux kernel
2.6.18 and GNU Compiler Collection (GCC version 4.1.2). The test
dataset is from Barski 2007
(\href{http://dir.nhlbi.nih.gov/papers/lmi/epigenomes/hgtcell.aspx}{link})
and Kairong 2009
(\href{http://dir.nhlbi.nih.gov/papers/lmi/epigenomes/hghscmethylation.aspx}{link}). In
particular, in the analysis of a single test sample, we used H3K36me3
data in human CD4+ T cells; in the analysis of a test sample and a
control sample, we used H3K36me3 data and anti-H3 data in human T
cells; finally in the analysis of two test samples, we used the
H3K36me3 data in human CD36+ erythrocyte precursor cells and human
CD133+ stem cells. The exact running time and memory usage varies for
other histone modifications and datasets, however are similar to that
reported here.

 
\begin{table}[th]
  \centering
  \begin{tabular}{c c c c c c c }
    \hline
    analysis type & genome & binning size & iterations & training size
    & running time & memory usage\\
    \hline
    test sample & human & 1000bp & 20 & 2508851 & ~9min & ~1.0G \\
    test and control sample & human & 1000bp & 30 & ~180000 & ~22min & ~1.3G \\
    test and test sample & human & 1000bp & 30 & ~180000 & ~50min &
    ~1.4G \\
   \hline
  \end{tabular}
  \caption{Resources requirement of RSEG}
  \label{tab:format-bin-diff}
\end{table}

\section{FAQ}
\label{sec:faq}

\textbf{1. [GSL] 
When I run the rseg command, it gives  the following error message: ./rseg:
error while loading shared libraries: libgsl.so.0: cannot open  shared object
file: No such file or directory}


RSEG needs GSL (GNU Scientific Library).  When you see this error, it is likely
that gsl is not installed on your machine. You may need to manually install it
from \url{http://www.gnu.org/software/gsl/}. Alternatively, there are
pre-compiled gsl packages on major Linux distribution, such as SUSE or UBUNTU.

\end{document}


